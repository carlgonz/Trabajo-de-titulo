\chapter{Dise�o del software}

\section{Resumen}

En este cap�tulo se describe el proceso de dise�o del software de control para el sat�lite. El proceso de dise�o considera en primera instancia los requerimientos funcionales de la misi�n as�, requerimientos generales de desarrollo, as� como la plataforma de hardware sobre la que se debe dise�ar para tener claro las limitaciones y alcances del dise�o final.

El dise�o del la aplicaci�n se detalla en diferentes niveles, incluyendo una visi�n global en base a m�dulo y se centra en espec�fico en la aplicaci�n de un determinado patr�n de dise�o que gu�a la soluci�n propuesta en forma de una arquitectura basada en \textit{command patern}

Para el dise�o de la arquitectura se ha utilizado un patr�n de dise�o llamado \textit{command pattern} el cual se programa en lenguaje C para el compilador Microchip MPLAB C30 V3.3x utilizando el sistema operativo FreeRTOS. Todo esto sobre una plataforma de desarrollo CubesatKit que cuenta con un procesador PIC24FJ256GA110.

\section{Requerimientos}
\subsection{Requerimientos generales}
\subsection{Requerimientos operacionales}
Los requerimientos operacionales se refieren a las funcionalidades que se espera que el computador a bordo del bus SUCHAI deba realizar. Estos requerimientos son los requisitos b�sicos que el sistema debe cumplir para considerar que se cuenta con un sat�lite capaz de llevar a cabo la misi�n del proyecto SUCHAI. Estos requerimientos son agrupados por �rea para un mejor ****

\subsubsection{�rea de comunicaciones}
\paragraph{Configuraci�n inicial del \textit{transcevier}}
\paragraph{Despliegue de antenas}
\paragraph{Procesamiento de telecomandos}
\paragraph{Protocolo de enlace}
\paragraph{Env�o de telemetr�a}


\subsubsection{Control central}
\paragraph{Organizar telemetr�a}
\paragraph{Programa de vuelo}
\paragraph{Obtener el estado del sistema}
\paragraph{Tolerancia a fallos de software}
\paragraph{Secuencia de inicio}
\paragraph{Inicializaci�n del sistema}

\subsubsection{�rea de energ�a}
\paragraph{Estimaci�n de la carga de la bater�a}
\paragraph{Presupuesto de energ�a}

\subsubsection{�rbita}
\paragraph{Actualizar par�metros de �rbita}

\subsubsection{\textit{Payloads}}
\paragraph{Ejecuci�n de comandos de \textit{payloads}}

\subsubsection{Tolerancia a fallos}
\paragraph{Estado de salud del sistema}
\paragraph{Mucho tiempo sin conexi�n a tierra}
\paragraph{Problemas al desplegar la antena}
\paragraph{\textit{Watchdog}}
\paragraph{Fallos de hardware}

\section{Plataforma}
Se describe la plataforma final donde se ejecutar� la aplicaci�n, en este caso un sistema embebido consistente de un PIC24FJ256GA110

\section{Arquitectura de software}
\subsection{Arquitectura Global}
\subsection{Controladores de hardware}
\subsection{Sistema operativo}
\subsection{Aplicaci�n}