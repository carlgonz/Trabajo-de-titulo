\documentclass[11pt,letterpaper]{article}
\usepackage[spanish]{babel}
%\usepackage[ansinew]{inputenc}
\usepackage[utf8]{inputenc}
%\usepackage[latin1]{inputenc}
\usepackage[letterpaper,includeheadfoot, top=0.5cm, bottom=3.0cm, right=2.0cm, left=2.0cm]{geometry}
\renewcommand{\familydefault}{\sfdefault}

\usepackage{graphicx}
\usepackage{color}
\usepackage{hyperref}
\usepackage{amssymb}
\usepackage{url}
%\usepackage{pdfpages}
\usepackage{fancyhdr}
\usepackage{hyperref}
\usepackage{subfig}

\usepackage{listings} %Codigo
\lstset{language=C, tabsize=4,framexleftmargin=5mm,breaklines=true}

\begin{document}
% --------------- ---------PORTADA --------------------------------------------
\newpage
\pagestyle{fancy}
\fancyhf{}
%-------------------- CABECERA ---------------------
\fancyhead[L]{ \includegraphics[scale=0.9]{img/logo_die.pdf} }
%------------------ TÍTULO -----------------------
\vspace*{6cm}
\begin{center}
\Huge  {Contextualización} \\
\vspace{1cm}
\huge {EL6908 - Introducción al Trabajo de Título}\\
\vspace{1cm}
\huge {\textit{Diseño e Implementación del Software de Control para el Computador a Bordo de un Pico-Satélite}}\\
%\vspace{1cm}
%\small {Título pequeño} 
\end{center}
%----------------- NOMBRES ------------------------
\vfill
\begin{flushright}
\begin{tabular}{ll}
\textbf{Autor} &: Carlos González C.\\
\textbf{Profesor Guía} &: Marcos Díaz Q.\\
\textbf{Profesor EL6908} &: Jorge Lopez H.\\
& \today\\
& Santiago, Chile.
\end{tabular}
\end{flushright}

% ·············· ENCABEZADO ············
\newpage
\pagestyle{fancy}
\fancyhf{}
%\fancyhead[L]{\rightmark}
\fancyhead[L]{\small \rm \textit{Sección \rightmark}}
\fancyhead[R]{\small \rm \textbf{\thepage}}
\fancyfoot[L]{\small \rm \textit{Contextualización}}
\fancyfoot[R]{\small \rm \textit{EL6908 - Introducción al Trabajo de Título}}
%\fancyfoot[C]{\thepage}
\renewcommand{\sectionmark}[1]{\markright{\thesection.\ #1}}
\renewcommand{\headrulewidth}{0.5pt}
\renewcommand{\footrulewidth}{0.5pt}

% =============== INDICE ===============

\tableofcontents
\listoffigures
\listoftables

% =============== CUERPO ===============
\newpage

\section{Contextualización}

En este capítulo se discuten los aspectos teóricos relacionados con el desarrollo de un proyecto de software para sistemas embebidos pensados en ser utilizados en misiones aeroespaciales, en específico el control del computador a bordo de un pico-satélite.

\subsection{Sistemas embebidos}

Los sistemas embebidos a diferencia de un computador personal que es usado con fines generales para una amplia variedad de tareas, son sistemas computacionales normalmente utilizados para atender una cantidad limitada de procesos, realizar tareas específicas o dotar de determinada inteligencia a un sistema más complejo. Un sistema embebido está compuesto por uno o más microcontroladores pequeños que cuentan con periféricos para manejar diferentes protocolos de comunicación; conversores ADC; timers; puertos de entrada y salida digitales, todo en integrado en un mismo chip para guardar espacio y ahorrar energía. Parte fundamental de un sistema embebido es el software que provee la funcionalidad final, usualmente se usa el término firmware para referirse a este código con que se programa el microcontrolador el cual por lo general es específico para la plataforma de hardware y se relaciona a muy bajo nivel. A diferencia de un computador de propósito general donde el usuario puede cargar una serie de programas en él para un amplio rango de usos, el usuario de un sistema embebido no tiene la capacidad de reprogramarlo fuera de las posibilidades que el desarrollador ha brindado al sistema\cite{EMBEDDED}.\\

Para el diseño de sistemas embebidos se debe considerar ciertos aspectos que los diferencian de otros tipos de sistemas de computacionales, tales como\cite{SE}:

\begin{itemize}
	\item Un sistema embebido se mantiene siempre funcionando y debe proveer respuesta en tiempo real. Se debe diseñar considerando una operación continua y una posible reconfiguración del sistema estando ya en marcha.
	
	\item Las interacciones con el sistema pueden ser impredecibles y no se tiene control sobre ellas. Existen sistemas que son controlados por el usuario mediante una interfaz preparada para ellos, mientras que otros sistemas deben atender eventos imprevistos sin dejar de realizar tareas rutinarias.
	
	\item Existen limitaciones físicas. Normalmente estos sistemas poseen limitadas características de: poder de cómputo, memoria de datos y de programa; espacio físico; y disponibilidad de energía.
	
	\item El diseño de software para sistemas embebidos requiere una interacción de bajo nivel. Existe una amplia gama de plataformas de hardware para desarrollar sistemas embebidos y se requiere interactuar también con una amplia gama de dispositivos externos. Por esto se requiere desarrollar capas de drivers de periféricos que oculten las diferencias de hardware a la aplicación final del sistema.
	
	\item Es importante considerar aspectos de seguridad y confiabilidad del sistema durante todo su desarrollo debido a que la mayoría de los sistemas embebidos son usados para controlar otros sistemas críticos en diversos procesos.
\end{itemize}

\subsubsection{Microcontroladores}

Todo sistema embebido está formado fundamentalmente por un microcontrolador que brinda la capacidad de cómputo y el control de diferentes periféricos que normalmente están integrados en el mismo chip. Entre los principales fabricantes de microcontroladores se encuentran: Microchip, Texas Instrument, ARM, Motorola, NVidia.


% ============= BIBLIOGRAFIA ==============
\newpage
\begin{thebibliography}{5}
	\bibitem{SE} I. Sommerville, \textit{Software Engineering}, 9nd ed., USA: Addison-Wesley, 2011.
	\bibitem{MPLAB} Microchip, \textit{MPLABX IDE User's Guide}, USA:  Microchip Technology Incorporated, 2011.
	\bibitem{EMBEDDED} S. Heat, \textit{Embedded Systems Design}, 2nd ed., England: Newnes, 2003.
	
\end{thebibliography}

\end{document}

% % ················ IMAGEN ·················
% \begin{figure}[ht!]
% \centering
% \fbox{\includegraphics[scale=0.6]{img/flujo.png}}
% \caption{Flujo de caja anual}\label{flujo}
% \end{figure}
% %··········································

% % ················ IMAGEN ·················
% \begin{figure}[ht!] \centering
% \subfloat[Esquemático]{\includegraphics[scale=0.44]{img/seguidor.png}}
% \subfloat[Simulación]{\includegraphics[scale=0.45]{img/seguidor1.png}}
% \caption{Simulación como seguidor de voltaje}\label{seguidor}
% \end{figure}
% %··········································